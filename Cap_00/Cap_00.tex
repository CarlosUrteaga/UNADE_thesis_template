 \chapter{Introducción}\label{Cap_00}

\section{Descripción del problema}
Descripción de la problemática que resolverá la tesis, explicando la relevancia de poder resolverla, como surge, cuál es el motivo para resolverla, a quien beneficiará si se resuelve y demás aspectos concernientes al problema a resolver.

\section{Hipótesis}
Debe ser un enunciado de máximo un párrafo en donde debes escribirse con la siguiente estructura:
La primera parte es la descripción afirmativa de lo que hará tu tesis y finalmente lo que se conseguirá como resultado de aplicarla. Ejemplo: “Si diseño una metodología de implementación de una red segura con diversas técnicas de cifrado, mejoraré la seguridad integral de las empresas que la implementen”.

\section{Justificación}
Son los motivos por los cuales es viable el desarrollo de la tesis, ¿Por qué es relevante que se resuelva este problema?, ¿Que viabilidad tiene, en cuanto a implementación, realización, costo-beneficio, esfuerzo, etc.?, ¿A quién beneficiará? ¿Que aportará?, ¿Tendrá un costo menor de algo similar en el mercado?, ¿Realizará algo mejor que otro desarrollo similar?, etc.

\section{Objetivos}
Objetivo general. Se detalla de forma general que va a realizar el trabajo de tesis.
Objetivos específicos. Es el conjunto de objetivos intermedios que permitirán completar el objetivo general, por lo general se enlistan en orden cronológico y debe ser medible.

\section{Organización del documento de tesis}
Se detallan el contenido de los capítulos siguientes.