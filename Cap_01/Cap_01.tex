\chapter{Marco de referencia}\label{Cap_00}
\lettrine[lines=2,nindent=0pt]{E}{}l presente capítulo muestra un contexto de la inteligencia artificial y su impacto en el sector financiero. También se presenta el objetivo, alcance y metodología utilizada en este trabajo. Al final del capítulo se detalla la organización del documento.

\section{Marco Teórico}
Aproximadamente de 20 a 40 cuartillas. Son todos los temas que se deben de saber para comprender el desarrollo de la tesis, Ejemplo: SI la tesis es referente a “un algoritmo de Inteligencia artificial para la elaboración de nano robots que construyan computadoras cuánticas”, en el marco teórico debe de venir la teoría referente a: Algoritmos de inteligencia artificial, elaboración de nano robots y construcción de computadoras cuánticas.

\section{Estado del arte}
Aproximadamente de 20 a 40 cuartillas. En este apartado se investigan todos los desarrollos similares a la tesis que estas realizando, primero en tu país de origen y posteriormente en el mundo, la aportación de este apartado es darse cuenta si lo que estamos haciendo es algo que está desarrollando en nuestro país o en el mundo, que tan viable es, que tan original es, que tanto diferente estoy aportando, etc.
